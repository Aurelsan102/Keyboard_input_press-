\documentclass{article}
\usepackage{graphicx} % Required for inserting images

\title{Rapport de projet}
\author{NGUAZONG TSAFACK Aurel\\20P001\\AIA-4}
\date{February 2024}

\begin{document}

\maketitle
\section{Introduction}

Les périphériques d’interface humaine (HID) sont une définition de classe de périphérique qui remplace les connecteurs de style PS/2 par un pilote USB générique pour prendre en charge les périphériques HID tels que les claviers, les souris, les contrôleurs de jeu, etc. Alors ceux-ci ont une importance primordiale dans le contrôle de nos machines et il faudrait savoir les manipuler d'où le projet.

\section{Méthodologie de travail}
\subsection{Technologie utilisée}
Lors de la scession de programmation du logiciel, nous avons choisi d'utiliser le langage \textbf{python} avec le GUI \textbf{Pygame} dans l'IDE \textbf{Visual Studio Code}.
\subsection{Installation des librairies}
Concernant ce projet, nous devrons importer la bibliothèque pygame en utilisant la commande:\\
import pygame\\
\\
Dans le cas où il n'y a pas la librairie installer dans la machine, veuillez lancer cette commande dans votre terminal si vous êtes sous Windows:\\
pip install pygame\\
\\
\subsection{Code}
Suite à l'installation des librairies :\\
import pygame\\
from pygame.locals import *\\
\\

L'on doit configurer l'interface de pygame qui va recevoir le texte de la touche appuyée qui s'affiche selon: touche pressée :(touche pressée). Pour résumer simplement, l'on récupère grâce à la gestion des évènements de pygame qui enregistre la touche récupérée dans une variable:\\
touche = pygame.key.name(evenement.key)
\section{Problèmes rencontrées}
Lors de développement, nous avons rencontré un souci lors de la récupération de la touche qui débouchais sur une boucle infinie, mais résolu par une recherche sur chatgpt.

\end{document}
